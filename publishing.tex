
\chapter{Self Publishing}
\label{chapter:publishing}

\thispagestyle{empty}

\lipsum[2]


\minitoc
\newpage



% *******************************************
% SECTION
% ******************************************* 
\section{Getting an ISBN Number}


You need an ISBN number for your book. There are two choices. 
If you use Kindle Direct Publishing (KDP) as your publishing platform, KDP will give you an
ISBN number for free. The downside is that the ISBN number is tied to KDP (the first part of
the ISBN number contains an ID assigned to KDP), so if later you want to
publish the book using a second platform (you can freely do so for the same book), 
you cannot use this ISBN number. 

I published the first edition of my book using CreateSpace, which was later merged into KDP. I
did use the ISBN number assigned by it. In the second edition of my book. I bought my own ISBN
numbers, i.e., I become a publisher myself. I published my book simultaneously on two different
platforms (see the next section). 

There is only one place in the US to purchase ISBN numbers. It is Bowker. I strongly suggest
you buy 10 ISBNs, instead of 1. The price for 10 numbers is \$295, while the price for one is
\$125. For the 2nd edition of my book, I have used 4 numbers. If you want to publish your books
using a hardcover and also in paperback (the printing cost of hardcover is usually \$10 more), 
you need to use two ISBN numbers.  




% *******************************************
% SECTION
% ******************************************* 
\section{Choosing Platform}


There are several self-publishing platforms. I have read a lot of online comparisons, and
concluded that Kindle Direct Publishing (KDP) and IngramSpark are the two best choices. Each of
time has pros and cons, which I will explain next. To take advantage of their pros, many people
have suggested publishing simultaneously using both platforms. That is exactly what I did, and so far,
I am quite happy with this decision. 


Each platform basically uses a web interface to guide you through the entire procedure,
including choosing the trim size of your book, selecting the types of covers, submitting your
final PDF file, designing your cover, etc. The procedures are quite similar in both platforms. 
KDP's user interface is more user friendly. This is simply because KDP is owned by Amazon, and
maybe their developers did a better engineering job. However, as a computer-savy person, I 
don't have trouble navigating through IngramSpark's user interface.


\paragraph{Kindle Direct Publishing (KDP).}
The biggest advantage of KDP is the fact that it belongs to Amazon. I sell my books to
individual customers mostly from Amazon. So, if a book is published via KDP, it is listed on
Amazon as always available, and customers can get the book in 2 days if they have the prime
membership.  For book published via IngramSpark, although the book can also be
sold through Amazon, many times, I see my book listed as unavailable until certain dates
(ranging from a few days to a month). The extra waiting time may affect a customer's buy or
no-buy decision making.  
Therefore, if you want to sell via Amazon, using KDP is definitely a better choice. 


Other than selling through Amazon, KDP also offers expanded distribution channels. Book stores
will unlikely buy from Amazon; instead, they purchase books from other channels, where they can
get discount. KDP's distribution channels are mainly for that purpose. 
This is where KDP sucks and IngramSpark shines. 


% -------------------------------------------
% SUBSECTION
% ------------------------------------------- 
\subsection{IngramSpark.} 
KDP forces you to give a very huge discount for the expanded distribution channels. I don't
like it. If a bookstore is a brick-and-mortar store, it deserves to enjoy the huge discount,
because it has to take a risk carrying your books in its inventories. However, many of the
bookstores who order from the distribution channels are not brick-and-mortar store; they
don't carry inventories; they are online stores. The discount offered by KDP is too much and
too unfair for authors. 
IngramSpark allows you to set your own discount rate (with some minimum). I did take advantage
of this feature and reduced the discount rate to a value that I feel more comfortable. 


Another advantage of IngramSpark is that I can get timely sales report from IngramSpark. For 
KDP, sales from Amazon can be reported very quickly, but sales through the expanded
distribution channels are reported with a long delay, and sometimes, strange things happened. 
One time, I didn't get any sale from
KDP's expanded distribution channels for 3 months. That was very abnormal based on the history
of the sales. After I contacted
KDP's customer service several times, sales started to show up, but I still don't know what happened to that
3 months' sales; did they get lost. I have since lost confidence in KDP's expanded distribution
channels, and decided to turn off those channels for good.


KDP does not offer the hardcover option (still don't know why they don't do this), but
IngramSpark does offer it. If you want to have a hardcover for your book, you can't use KDP.


I end up self publishing my books using both KDP and IngramSpark. For KDP, I only allowed the
Amazon's sales channel, and I disabled KDP's expanded distribution channels. 
Instead, for sales to book stores, I rely on IngramSpark. 
Being able to selling your books via non-Amazon channels is necessary, 
especially for textbooks, because 
university bookstores and libraries are not willing to order books from Amazon; 
they often order books from large book distributors, such as Ingram (IngramSpark is
related to Ingram). 





% *******************************************
% SECTION
% ******************************************* 
\section{Making Covers} 

Both KDP and IngramSpark provides template for your cover (including front, back, and spine). 
The template will be generated based on the trim size, type of cover, and the number of pages
in your book. Once you get the template, you can use Adobe Photoshop or other software to
make your own cover. You can also hire somebody to do that for you. 



% *******************************************
% SECTION
% ******************************************* 
\section{Uploading Your PDF File}


After you upload your book, the self-publishing platform will check the format of your book,
and reporting to you all the identified problems. There are two types of problems: serious
problems that must be fixed, and warnings that can be ignored. 


The most common problems I encountered is the PDF font problem. All the fonts must embedded in
the PDF file. If you see this problem, you can use Adobe Acrobat to embed all the fonts: click
the \texttt{Print} menu, and select \texttt{Adobe PDF}. Click the \texttt{Properties} button,
and in the \texttt{Adobe PDF Settings} tab, open the dropdown menu from
\texttt{"Default Settings"}, and select the \texttt{"High Quality Print"} option.
Then select the \texttt{Edit} button, click the \texttt{Fonts} tab. Make sure that 
the \texttt{"Embed all fonts"} option is selected.  


If you have fixed all the required problems, you can resubmit. If everything is ok, you can
continue on to the next step. 




% *******************************************
% SECTION
% ******************************************* 
\section{Self-Publishing In Other Countries}

Self-publishing in other countries may not be allowed, 
I only know a little bit about China from the Internet.



\subsection{Working with a Publisher}

To publish your book in another country, you may need to work
with a publisher if self-publishing is not allowed. When you sign
the contract, you need to remember that you are not only representing the author, 
you are also representing a
publisher (which is yourself). Most publishers may not realize that they 
are dealing with a publisher, so their contracts may not correctly reflect the fact. 
For example, the contract may ask you to grant them exclusive world-wide publishing rights. 
If you grant that, you may lose your self-publishing right.  
When I negotiated with a publisher in China, I have requested to add two restrictions in the 
contract: (1) the publisher can only publish the book in the Chinese language, 
and it does not have rights to publish the book in other languages. 
(2) The publisher can only publish the book in China. 
The publisher did accept my term. 


I am not a lawyer or expert on this issue, so do not treat my suggestions as 
legal advices. They are based on my common sense, so 
feel free to send me corrections if they are incorrect. 





