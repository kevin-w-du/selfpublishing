
\chapter{Latex Tricks}
\label{chapter:tricks}

\thispagestyle{empty}

I have learned and figured out some tricks that I find very useful. 


\minitoc
\newpage




% *******************************************
% SECTION
% ******************************************* 
\section{Avoiding Screenshots If Possible}

Screenshots are placed inside books as an image. One of the biggest
problems in Latex is that images are floating, i.e., if you want to place
the image below some paragraph, you may not be able to get that result.
If space is not enough, Latex may move the image to another page. That is
why I try to avoid using screenshots if possible.

One of the common screenshots is to capture the program
running results from a terminal window. Instead of using a screenshot, 
I copy and paste the text and put it inside a \texttt{lstlisting} block. 
See the following example: 

\begin{lstlisting}
$ chmod u+x exploit.py      (*@\reflectbox{\ding{217}} \textbf{make it executable}@*)
$ rm badfile
$ exploit.py
$ ./stack
# id       (*@\reflectbox{\ding{217}} \textbf{Got the root shell!}@*)
uid=1000(seed) gid=1000(seed) (*@\textbf{euid=0(root)}@*) groups=0(root), ...
\end{lstlisting}

The example above also shows how easy it is to annotate the
``screenshot''. If this were an image, annotation will be much
more complicated. Moreover, if you want to make some changes, such as 
changing the name of the program, you can directly modify the text. If this
were an image, you have to retake the screenshot. Sometimes, it may
take you a long time to set up the environment so you can take
the screenshot. 



% *******************************************
% SECTION
% ******************************************* 
\section{Generating Figures-Only PDF}


If you write a textbook, you may want to make all your figures available to
instructors, so they can use the figures to make their slides. In the first
edition of my book, I did it manually, and it took me quite a long time. 
In the second edition, I learned a
trick. I simply added the following to my Latex file (the main file). 
When I use Latex to compile the main file, Latex will extract all the 
figures from the book, and place them inside a new file 
called \texttt{bookfigureonly\_new.tex} (this name is 
specified by me in the following and it can be any name).




\begin{lstlisting}
\usepackage[
active,
generate=bookfigureonly_new,
copydocumentclass=false,
extract-cmd=chapter,
extract-env={figure}
]{extract}
\begin{extract}
\documentclass[10pt]{book}

% Use the following set up, it works for CreateSpace and IngramSpark. 
\usepackage{geometry}
\geometry{paperwidth=7.5in,paperheight=9.25in, bindingoffset=2cm,
          left=1.00in,right=0.50in,top=0.75in,bottom=0.50in,twoside}


\usepackage[most]{tcolorbox}
\usepackage{times}
\usepackage{epsf}
\usepackage{epsfig}
\usepackage{amsmath, alltt, amssymb, xspace}
\usepackage{wrapfig}
\usepackage{fancyhdr}
\usepackage{url}
\usepackage{verbatim}
\usepackage{fancyvrb}
\usepackage{listings}
\usepackage{color}
\usepackage{minitoc}
\usepackage{makeidx}
\usepackage{lipsum}


\usepackage{subfigure}
%\usepackage[style=numeric]{biblatex}
%\usepackage{cite}
\usepackage[square]{natbib}
\usepackage{hyperref}
\hypersetup{%
    pdfborder = {0 0 0}
}

\usepackage{sidecap}
\usepackage{pifont}
\usepackage{mdframed}
\usepackage{textcomp}
\usepackage{comment}
\usepackage{enumitem}

% for watermark 
%\usepackage{draftwatermark}

\newcommand{\ubuntutwo}{{\tt Ubuntu12.04}\xspace}
\newcommand{\ubuntusix}{{\tt Ubuntu16.04}\xspace}

\newcommand{\todo}[1]{
\vspace{0.1in}
\fbox{\parbox{0.9\textwidth}{TODO: #1}}
\vspace{0.1in}
}

\newcommand{\mybox}[1]{
\vspace{0.2in}
\noindent
\fbox{\parbox{6.5in}{#1}}
\vspace{0.1in}
}

\definecolor{dkgreen}{rgb}{0,0.6,0}
\definecolor{gray}{rgb}{0.5,0.5,0.5}
\definecolor{mauve}{rgb}{0.58,0,0.82}
\definecolor{lightgray}{gray}{0.90}


\lstset{%
  frame=none,
  language=,
  backgroundcolor=\color{lightgray}, 
  aboveskip=3mm,
  belowskip=3mm,
  showstringspaces=false,
%  columns=flexible,
  basicstyle={\small\ttfamily},
  numbers=none,
  numberstyle=\tiny\color{gray},
  keywordstyle=\color{blue},
  commentstyle=\color{dkgreen},
  stringstyle=\color{mauve},
  breaklines=true,
  breakatwhitespace=true,
  tabsize=3,
  columns=fullflexible,
  keepspaces=true,
  escapeinside={(*@}{@*)}
}


% This should be turned on when we compile the whole book
% Control the depth of the minitoc. The default is 2, but 
% I prefer to use 1.
\setcounter{minitocdepth}{1}
\setcounter{parttocdepth}{0}


% Add chapter number to enumerate list number
% Each chapter will redefine this \mychapternumber variable.
%\newcommand{\mychapternumber}{0}
%\setenumerate[1]{label=\mychapternumber.\arabic*.}
%\setenumerate[1]{label=\thechapter.\arabic*.}
%\setenumerate[2]{label=\arabic*.}


\newcommand{\seedurl}{\url{https://seedsecuritylabs.org}}
\newcommand{\problemurl}{\url{https://www.handsonsecurity.net/}}
\newcommand{\bookurl}{\url{https://www.handsonsecurity.net/}}



\usepackage{microtype}

\tcbuselibrary{skins}
\usepackage{lipsum}
\usepackage{tikz}

\newcommand*\circled[1]{\tikz[baseline=(char.base)]{
                        \node[text=white,text centered,shape=circle,draw,fill=black,
                                inner sep=0.2pt,minimum size=1pt] (char)
             {\footnotesize #1};}}


\makeatletter
\newcommand{\manuallabel}[2]{\def\@currentlabel{#2}\label{#1}}
\makeatother


% This the one that we will use
\newtcolorbox{myfigbox}[2]{colback=blue!5!white,colframe=blue!75!black,
              title=\figurename~#1: #2}
\def \mylabel#1{\stepcounter{figure} \manuallabel{#1}{\thefigure}}


% Default is on top
\newtcolorbox{mysidebar}[1][t]
%{float=#1, colback=blue!5!white,colframe=blue!75!black, title=#2, enhanced}
{ 
   enhanced,
   title = {Sidebar \thetcbcounter},
   float=#1,  
}


\newtcolorbox[auto counter,number within=chapter]{mysidebar2}[1][]
{  
   enhanced,
   fonttitle=\scshape,
   title = {Sidebar \thetcbcounter},
   float=#1
}


% Default is on the right side
\newenvironment{mywrapsidebar}[1][R]
  {\wrapfigure{#1}{0.5\textwidth}\tcolorbox}{\endtcolorbox\endwrapfigure}



%%%%%%%%%%%%%%%%%%%%%%%%%%%%%%%%%%%%%%%%%
% The above are being used
%------------------------------------------------------------------------------
% The below are still in the trial phase. 
%%%%%%%%%%%%%%%%%%%%%%%%%%%%%%%%%%%%%%%%%



% We will use this to include figures.
% Note: we increase the counter ourselves, so inside the figure, we can have caption 
% any more, because \caption automatically increases the counter.
% Without \caption, the \label will take the section number, instead of the 
%  figure number, so we can't use \label any more.
\newenvironment{myfigure}[2]
{
    \begin{figure}[#1]
    \stepcounter{figure}
    \begin{tcolorbox}[colback=blue!5!white,colframe=blue!75!black,
       title=\figurename~\thefigure: #2]
}
{\end{tcolorbox}\end{figure}}


% This version is not used any more.
\newenvironment{mymyfigure}[1]
{
    \stepcounter{figure}
    \begin{tcolorbox}[colback=blue!5!white,colframe=blue!75!black,
       title=\figurename~\thefigure: #1]
}
{\end{tcolorbox}}




\newenvironment{mywrapfigure}[1]
{
    \stepcounter{figure}
    \begin{tcolorbox}[colback=blue!5!white,colframe=blue!75!black,
       title=\figurename~\thefigure: #1]
}
{\end{tcolorbox}}





%\newtcolorbox[blend into=figures]{myfigure}[2][]{float=htb,capture=hbox,
%title={#2},every float=\centering,#1}


%Defining a newtcolorbox to be a standard box for all examples in the text:
\newtcolorbox[auto counter,number within=section]{phbox}[1][]{skin=bicolor,
                title=Figure~\thefigure,#1,
                size=title,colback=white,colbacklower=black!10!white}




\newenvironment{tcfigure}[1]
{\stepcounter{figure}%
\tcbset{colback=blue!5!white,colframe=blue!75!black,title=\figurename~\thefigure: #1}
\begin{tcolorbox}}
{\end{tcolorbox}}



\tcbset{frogbox/.style={enhanced,colback=green!10,colframe=green!65!black,
		enlarge top by=5.5mm,
		overlay={\foreach \x in {2cm,3.5cm} {
				\begin{scope}[shift={([xshift=\x]frame.north
					west)}]
					\path[draw=green!65!black,fill=green!10,line
					width=1mm] (0,0) arc (0:180:5mm);
					\path[fill=black] (-0.2,0) arc
					(0:180:1mm);
\end{scope}}}}}


\usetikzlibrary{patterns} % preamble
% \tcbuselibrary{skins} % preamble
\tcbset{ribbonbox/.style={enhanced,colback=red!5!white,colframe=red!75!black,
		fonttitle=\bfseries,
		overlay={\path[fill=blue!75!white,draw=blue,double=white!85!blue,
				preaction={opacity=0.6,fill=blue!75!white},
				line width=0.1mm,double distance=0.2mm,
				pattern=fivepointed stars,pattern
			color=white!75!blue]
			([xshift=-0.2mm,yshift=-1.02cm]frame.north east)
			-- ++(-1,1) -- ++(-0.5,0) -- ++(1.5,-1.5) --
cycle;}}}


%\newtcolorbox[blend into=figures]{myfigure}[2][]{float=htb,capture=hbox,
%title={#2},every float=\centering,#1}


\usepackage{varwidth}
\newtcolorbox{mynewbox}[2][]{enhanced,
	attach boxed title to top left={xshift=1cm,yshift=-2mm},
	fonttitle=\bfseries,varwidth boxed title=0.7\linewidth,
	colbacktitle=green!45!white,coltitle=green!10!black,colframe=green!50!black,
	interior style={top color=yellow!10!white,bottom
	color=green!10!white},
	boxed title style={enhanced,boxrule=0.75mm,colframe=white,
		borderline={0.1mm}{0mm}{green!50!black},
		borderline={0.1mm}{0.75mm}{green!50!black},
		interior style={top color=green!10!white,bottom
			color=green!10!white,
		middle color=green!50!white},
	drop fuzzy shadow}, title={#2},#1}



\usetikzlibrary{decorations.text,decorations.pathreplacing,calc}

\newcommand{\tikzmark}[2]{%
     \tikz[overlay,remember picture] \node[text=black, inner sep=2pt] (#1) {#2};}


\newcommand*{\AddNote}[4]{%
    \begin{tikzpicture}[overlay, remember picture]
        \draw [decoration={brace,amplitude=0.5em},decorate,ultra thick,black]
            ($(#3)!([yshift=1.5ex]#1)!($(#3)-(0,1)$)$) --
            ($(#3)!(#2)!($(#3)-(0,1)$)$)
                node [align=center, text width=3.5cm, pos=0.5, anchor=west] {#4};
    \end{tikzpicture}
}





\input{figure_folder}
\end{extract}
\end{lstlisting}

After you get the figure-only tex file, you can run 
Latex on this file. You will get a PDF file that contains
all your figures. 
 

% *******************************************
% SECTION
% ******************************************* 
\section{Sharing Code Using GitHub}

For code used in my book, I put them in a public GitHub repository, so
readers can easily download my code.

