\documentclass[10pt]{book}

% Use the following set up, it works for CreateSpace and IngramSpark. 
\usepackage{geometry}
\geometry{paperwidth=7.5in,paperheight=9.25in, bindingoffset=2cm,
          left=1.00in,right=0.50in,top=0.75in,bottom=0.50in,twoside}


\usepackage[most]{tcolorbox}
\usepackage{times}
\usepackage{epsf}
\usepackage{epsfig}
\usepackage{amsmath, alltt, amssymb, xspace}
\usepackage{wrapfig}
\usepackage{fancyhdr}
\usepackage{url}
\usepackage{verbatim}
\usepackage{fancyvrb}
\usepackage{listings}
\usepackage{color}
\usepackage{minitoc}
\usepackage{makeidx}
\usepackage{lipsum}


\usepackage{subfigure}
%\usepackage[style=numeric]{biblatex}
%\usepackage{cite}
\usepackage[square]{natbib}
\usepackage{hyperref}
\hypersetup{%
    pdfborder = {0 0 0}
}

\usepackage{sidecap}
\usepackage{pifont}
\usepackage{mdframed}
\usepackage{textcomp}
\usepackage{comment}
\usepackage{enumitem}

% for watermark 
%\usepackage{draftwatermark}

\newcommand{\ubuntutwo}{{\tt Ubuntu12.04}\xspace}
\newcommand{\ubuntusix}{{\tt Ubuntu16.04}\xspace}

\newcommand{\todo}[1]{
\vspace{0.1in}
\fbox{\parbox{0.9\textwidth}{TODO: #1}}
\vspace{0.1in}
}

\newcommand{\mybox}[1]{
\vspace{0.2in}
\noindent
\fbox{\parbox{6.5in}{#1}}
\vspace{0.1in}
}

\definecolor{dkgreen}{rgb}{0,0.6,0}
\definecolor{gray}{rgb}{0.5,0.5,0.5}
\definecolor{mauve}{rgb}{0.58,0,0.82}
\definecolor{lightgray}{gray}{0.90}


\lstset{%
  frame=none,
  language=,
  backgroundcolor=\color{lightgray}, 
  aboveskip=3mm,
  belowskip=3mm,
  showstringspaces=false,
%  columns=flexible,
  basicstyle={\small\ttfamily},
  numbers=none,
  numberstyle=\tiny\color{gray},
  keywordstyle=\color{blue},
  commentstyle=\color{dkgreen},
  stringstyle=\color{mauve},
  breaklines=true,
  breakatwhitespace=true,
  tabsize=3,
  columns=fullflexible,
  keepspaces=true,
  escapeinside={(*@}{@*)}
}


% This should be turned on when we compile the whole book
% Control the depth of the minitoc. The default is 2, but 
% I prefer to use 1.
\setcounter{minitocdepth}{1}
\setcounter{parttocdepth}{0}


% Add chapter number to enumerate list number
% Each chapter will redefine this \mychapternumber variable.
%\newcommand{\mychapternumber}{0}
%\setenumerate[1]{label=\mychapternumber.\arabic*.}
%\setenumerate[1]{label=\thechapter.\arabic*.}
%\setenumerate[2]{label=\arabic*.}


\newcommand{\seedurl}{\url{https://seedsecuritylabs.org}}
\newcommand{\problemurl}{\url{https://www.handsonsecurity.net/}}
\newcommand{\bookurl}{\url{https://www.handsonsecurity.net/}}

