
\usepackage{microtype}

\tcbuselibrary{skins}
\usepackage{lipsum}
\usepackage{tikz}

\newcommand*\circled[1]{\tikz[baseline=(char.base)]{
                        \node[text=white,text centered,shape=circle,draw,fill=black,
                                inner sep=0.2pt,minimum size=1pt] (char)
             {\footnotesize #1};}}


\makeatletter
\newcommand{\manuallabel}[2]{\def\@currentlabel{#2}\label{#1}}
\makeatother


% This the one that we will use
\newtcolorbox{myfigbox}[2]{colback=blue!5!white,colframe=blue!75!black,
              title=\figurename~#1: #2}
\def \mylabel#1{\stepcounter{figure} \manuallabel{#1}{\thefigure}}


% Default is on top
\newtcolorbox{mysidebar}[1][t]
%{float=#1, colback=blue!5!white,colframe=blue!75!black, title=#2, enhanced}
{ 
   enhanced,
   title = {Sidebar \thetcbcounter},
   float=#1,  
}


\newtcolorbox[auto counter,number within=chapter]{mysidebar2}[1][]
{  
   enhanced,
   fonttitle=\scshape,
   title = {Sidebar \thetcbcounter},
   float=#1
}


% Default is on the right side
\newenvironment{mywrapsidebar}[1][R]
  {\wrapfigure{#1}{0.5\textwidth}\tcolorbox}{\endtcolorbox\endwrapfigure}



%%%%%%%%%%%%%%%%%%%%%%%%%%%%%%%%%%%%%%%%%
% The above are being used
%------------------------------------------------------------------------------
% The below are still in the trial phase. 
%%%%%%%%%%%%%%%%%%%%%%%%%%%%%%%%%%%%%%%%%



% We will use this to include figures.
% Note: we increase the counter ourselves, so inside the figure, we can have caption 
% any more, because \caption automatically increases the counter.
% Without \caption, the \label will take the section number, instead of the 
%  figure number, so we can't use \label any more.
\newenvironment{myfigure}[2]
{
    \begin{figure}[#1]
    \stepcounter{figure}
    \begin{tcolorbox}[colback=blue!5!white,colframe=blue!75!black,
       title=\figurename~\thefigure: #2]
}
{\end{tcolorbox}\end{figure}}


% This version is not used any more.
\newenvironment{mymyfigure}[1]
{
    \stepcounter{figure}
    \begin{tcolorbox}[colback=blue!5!white,colframe=blue!75!black,
       title=\figurename~\thefigure: #1]
}
{\end{tcolorbox}}




\newenvironment{mywrapfigure}[1]
{
    \stepcounter{figure}
    \begin{tcolorbox}[colback=blue!5!white,colframe=blue!75!black,
       title=\figurename~\thefigure: #1]
}
{\end{tcolorbox}}





%\newtcolorbox[blend into=figures]{myfigure}[2][]{float=htb,capture=hbox,
%title={#2},every float=\centering,#1}


%Defining a newtcolorbox to be a standard box for all examples in the text:
\newtcolorbox[auto counter,number within=section]{phbox}[1][]{skin=bicolor,
                title=Figure~\thefigure,#1,
                size=title,colback=white,colbacklower=black!10!white}




\newenvironment{tcfigure}[1]
{\stepcounter{figure}%
\tcbset{colback=blue!5!white,colframe=blue!75!black,title=\figurename~\thefigure: #1}
\begin{tcolorbox}}
{\end{tcolorbox}}



\tcbset{frogbox/.style={enhanced,colback=green!10,colframe=green!65!black,
		enlarge top by=5.5mm,
		overlay={\foreach \x in {2cm,3.5cm} {
				\begin{scope}[shift={([xshift=\x]frame.north
					west)}]
					\path[draw=green!65!black,fill=green!10,line
					width=1mm] (0,0) arc (0:180:5mm);
					\path[fill=black] (-0.2,0) arc
					(0:180:1mm);
\end{scope}}}}}


\usetikzlibrary{patterns} % preamble
% \tcbuselibrary{skins} % preamble
\tcbset{ribbonbox/.style={enhanced,colback=red!5!white,colframe=red!75!black,
		fonttitle=\bfseries,
		overlay={\path[fill=blue!75!white,draw=blue,double=white!85!blue,
				preaction={opacity=0.6,fill=blue!75!white},
				line width=0.1mm,double distance=0.2mm,
				pattern=fivepointed stars,pattern
			color=white!75!blue]
			([xshift=-0.2mm,yshift=-1.02cm]frame.north east)
			-- ++(-1,1) -- ++(-0.5,0) -- ++(1.5,-1.5) --
cycle;}}}


%\newtcolorbox[blend into=figures]{myfigure}[2][]{float=htb,capture=hbox,
%title={#2},every float=\centering,#1}


\usepackage{varwidth}
\newtcolorbox{mynewbox}[2][]{enhanced,
	attach boxed title to top left={xshift=1cm,yshift=-2mm},
	fonttitle=\bfseries,varwidth boxed title=0.7\linewidth,
	colbacktitle=green!45!white,coltitle=green!10!black,colframe=green!50!black,
	interior style={top color=yellow!10!white,bottom
	color=green!10!white},
	boxed title style={enhanced,boxrule=0.75mm,colframe=white,
		borderline={0.1mm}{0mm}{green!50!black},
		borderline={0.1mm}{0.75mm}{green!50!black},
		interior style={top color=green!10!white,bottom
			color=green!10!white,
		middle color=green!50!white},
	drop fuzzy shadow}, title={#2},#1}



\usetikzlibrary{decorations.text,decorations.pathreplacing,calc}

\newcommand{\tikzmark}[2]{%
     \tikz[overlay,remember picture] \node[text=black, inner sep=2pt] (#1) {#2};}


\newcommand*{\AddNote}[4]{%
    \begin{tikzpicture}[overlay, remember picture]
        \draw [decoration={brace,amplitude=0.5em},decorate,ultra thick,black]
            ($(#3)!([yshift=1.5ex]#1)!($(#3)-(0,1)$)$) --
            ($(#3)!(#2)!($(#3)-(0,1)$)$)
                node [align=center, text width=3.5cm, pos=0.5, anchor=west] {#4};
    \end{tikzpicture}
}




